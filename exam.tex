\RequirePackage{xcolor}
\documentclass{report}
% Change "article" to "report" to get rid of page number on title page
% \usepackage{amsmath,amsfonts,amsthm,amssymb}
\usepackage{setspace}
\usepackage{fancyhdr}
\usepackage{lastpage}
\usepackage{extramarks}
\usepackage{textcomp}
\usepackage{amsmath}
\usepackage{lstcustom}
\usepackage{enumitem}
\usepackage{multicol}
\usepackage{url}
\usepackage{tikz}

\usepackage{totcount} % For the total points

\usepackage{xspace}

% In case you need to adjust margins:
\topmargin=-0.45in
\evensidemargin=0in
\oddsidemargin=0in
\textwidth=6.5in
\textheight=9.0in
\headsep=0.25in

% Commands for answer key
\newif\ifkey
%\keytrue
\keyfalse

\newcommand{\correct}[1]{\ifkey\color{red}\textbf{#1}\color{black}\else\textbf{#1}\fi\xspace}
\newcommand{\answershort}[1]{\ifkey\color{red}\underline{\textbf{#1}}\color{black}\else\underline{\hspace{3in}}\fi\xspace}
\newcommand{\answervshort}[1]{\ifkey\color{red}\underline{\textbf{#1}}\color{black}\else\underline{\hspace{1in}}\fi\xspace}
\newcommand{\answervvshort}[1]{\ifkey\color{red}\underline{\textbf{#1}}\color{black}\else\underline{\hspace{0.5in}}\fi\xspace}
\newcommand{\answerlong}[1]{\ifkey\color{red}\textbf{#1}\color{black}\else\vspace{0.5in}\fi\xspace}

\newcommand{\vshortpts}{29}
\newcommand{\shortpts}{37}
\newcommand{\dbqpts}{28}

% Commands for points
\newtotcounter{points}
\newcommand*{\totalpoints}{\thepoints}
\newcommand*{\pts}[1]{\addtocounter{points}{#1}(#1pt)}

\newtotcounter{ecpoints}
\newcommand*{\totalecpoints}{\theecpoints}
\newcommand{\ecpts}[1]{\addtocounter{ecpoints}{#1}(#1pt)}

% Exercise Specific Information
\newcommand{\hmwkTitle}{Final Exam}
\newcommand{\hmwkDueDate}{UCID:\underline{\hspace{1in}}}
\newcommand{\hmwkClass}{CS 490} \newcommand{\hmwkPoints}{
  \ref{lastQuestion} questions; \protect\total{points} pts +
  \protect{\total{ecpoints}} ec; \pageref{LastPage} pgs.}

\newcommand{\cstart}{\vspace{.4cm}}

% Setup the header and footer
\pagestyle{fancy}
\lhead{\hmwkClass\ \hmwkTitle} 
\chead{\hmwkPoints}
\rhead{\hmwkDueDate}
\lfoot{\lastxmark}  
\cfoot{}            
\rfoot{Page\ \thepage\ of\ \pageref{LastPage}}
\renewcommand\headrulewidth{0.4pt} 
\renewcommand\footrulewidth{0.4pt}

% Setup listing look

\lstset{
  language=C++,
  style=eclipse,
  showspaces=false, 
  numbers=left,
  frame=tb,
}

\begin{document}

\begin{spacing}{1.4}

\begin{enumerate}[leftmargin=*]
\item \pts{1} \textbf{Name:} \hrulefill

\textbf{INSTRUCTIONS:}
Carefully read each question, and write the answer in the space
provided.  If answers to free response questions are written obscurely,
zero credit will be awarded. The correct answer to a free response question
will never contain any significant words used in the question itself (i.e., ``crossword rules'').
You are permitted one 8.5x11 inch sheet of paper (double-sided)
containing notes; all other aids (other than your brain) are forbidden.
Questions may be brought to the instructor.

For \textbf{TRUE} or \textbf{FALSE} and multiple choice questions,
circle your answer.

On free response questions only, you will receive \textbf{20\%} credit
for any question which you leave blank (i.e., do not attempt to
answer). Do not waste your time or mine by making up an answer if you
do not know. (Note though that most questions offer partial credit, so
if you know part of the answer, it is almost always better to write something
rather than nothing.)

To get credit for this question, you must:
\begin{itemize}
\item Print your name (e.g., ``Martin Kellogg'') in the space provided on this page.
\item Print your UCID (e.g., ``mjk76'') in the space at the top of \textbf{each} page of the exam.
\end{itemize}

\vspace{3in}

Contents (blanks for graders only):
\begin{tabular}{ll}
\textbf{Writing your name on every page:} & \answervshort{1} / 1\\
\textbf{I. Reading Quiz Redux:} & \answervshort{5} / 5\\
\textbf{II. Very Short Answer:} & \answervshort{\vshortpts} / \vshortpts \\
\textbf{III. Short Answer:} & \answervshort{\shortpts} / \shortpts \\
\textbf{IV. DBQs:} & \answervshort{\dbqpts} / \dbqpts \\
\textbf{V. Extra Credit:} & \answervshort{5} / 0 \\
\textbf{Total:} & \answervshort{105} / 100 \\
\end{tabular}

\newpage

\textbf{I. Reading Quiz Redux (5pts)}

%% \item \pts{1}
%%   \textbf{Static Analysis, Part 2:} \textbf{TRUE} or
%%   \correct{FALSE}: to use the verifier, engineers were taught how to
%%   use a special, declarative programming language that was not similar
%%   to their regular development language (C). The author's ICSE paper
%%   reports on how easy it was to teach this language to C developers.

%% \item \pts{1}
%%   \textbf{Debugging, Part 2:} \correct{TRUE} or \textbf{FALSE}:
%%   delta debugging requires a test to prove that each circumstance is really failure inducing.

%% \item \pts{1}
%%   \textbf{Code-level Design:} Name an advantage of \texttt{black} over the other Python linters discussed in the Yelp whitepaper. (< 5 words)
%%   \\ \answershort{any of: opinionated; resolves errors automatically; consistency}

\item \pts{1}
  \textbf{Tech Debt, Part 1:} \textbf{TRUE} or \correct{FALSE}:
  all technical debt is the result of programmer laziness.

%% \item \pts{1}
%%   \textbf{Tech Debt, Part 2:}
%%   The author claims that most programmers, when asked about the system they’re working on, “think the old code is a mess”. He posits this is due to a “fundamental law of programming”. Which one?
%%   \\ \correct{A}\hspace{0.2in}reading code is harder than writing code
%%   \\ \textbf{B}\hspace{0.2in} the halting problem
%%   \\ \textbf{C}\hspace{0.2in} given enough eyeballs, all bugs are shallow

\item \pts{1}
  \textbf{Requirements, Part 2:}
  The author describes formal specifications as providing three main benefits. Which of the following is NOT one of those:
  \\ \textbf{A}\hspace{0.2in} It provides clear documentation of the system requirements, behavior, and properties.
  \\ \textbf{B}\hspace{0.2in} It clarifies your understanding of the system.
  \\ \textbf{C}\hspace{0.2in} It finds really subtle, dangerous bugs.
  \\ \correct{D}\hspace{0.2in}It makes writing the code quicker and easier.

\item \pts{1}
  \textbf{Free and Open Source Software:}
  The author claims that the term “free software” means:
  \\ \textbf{A}\hspace{0.2in} software you can get for zero price
  \\ \correct{B}\hspace{0.2in}software which gives the user certain freedoms
  \\ \textbf{C}\hspace{0.2in} software whose source code you can look at
  \\ \textbf{D}\hspace{0.2in} none of the above

%% \item \pts{1}
%%   \textbf{Software Architecture, Part 1:}
%%   The author argues that which of the following should drive the design of a software system’s architecture:
%%   \\ \textbf{A}\hspace{0.2in} the existing implementation
%%   \\ \textbf{B}\hspace{0.2in} a set of guidelines from an architecture book
%%   \\ \correct{C}\hspace{0.2in}the system’s quality requirements

\item \pts{1}
  \textbf{Static Analysis, Part 1:}
  FindBugs \underline{\hspace{1in}}:
  \\ \textbf{A}\hspace{0.2in} always warns about line X if it is possible there is a bug on line X
  \\ \textbf{B}\hspace{0.2in} never warns about line X unless there is definitely a bug on line X
  \\ \textbf{C}\hspace{0.2in} both A and B
  \\ \correct{D}\hspace{0.2in}neither A nor B

\item \pts{1}
  \textbf{DevOps, Part 2:}
  Which of the following does Dan Luu advocate for when making a high-risk change?
  \\ \textbf{A}\hspace{0.2in} having multiple people watch or confirm the operation
  \\ \textbf{B}\hspace{0.2in} having ops people standing by in case of disaster
  \\ \correct{C}\hspace{0.2in}automating the change instead of letting a human do it
 

  \newpage

  \textbf{II. Multiple Choice and Very Short Answer (\vshortpts pts).} In the following section, either circle your
  answer (possible answers appear in \textbf{bold}) or write a very short (one word or one phrase) answer in the space provided.

\item \pts{2} Google (and other big tech companies) design their hiring process
  to avoid false \correct{positive} / \textbf{negative} results: that is, to avoid hiring unqualified candidates,
  even if some good candidates are rejected.

%% \item \pts{2} A \textbf{sound} / \correct{complete} program analysis always answers ``I don't know'' unless there
%%   is definitely a bug in the program being analyzed.

\item \pts{2} A \correct{functional specification} / \textbf{quality requirement} is a description of what a system should do that doesn’t specify how the system should do it.

\item \pts{2} Alice's deadline to deliver a feature is today. She finishes writing the feature and tests it on her local machine, but chooses not to write automated tests,
  even though she knows that it is risky, because
  she wants to meet her deadline. This is an example of \answershort{technical debt}

\item \pts{2}
  Which of the following is NOT a static analysis:
  \\ \textbf{A}\hspace{0.2in} dataflow analysis
  \\ \correct{B}\hspace{0.2in}testing
  \\ \textbf{C}\hspace{0.2in} code review

\item \pts{2}
  Which of the following is it best practice to commit to your version control system? Circle all that apply.
  \\ \textbf{A}\hspace{0.2in} credentials
  \\ \correct{B}\hspace{0.2in}code
  \\ \textbf{C}\hspace{0.2in} binary files
  \\ \correct{D}\hspace{0.2in}config files

%% \item \pts{2}
%%   Which of the following could make a good milestone in a software project? Circle all that apply.
%%   \\ \textbf{A}\hspace{0.2in} the code is ``50\% done''
%%   \\ \correct{B}\hspace{0.2in}a user story
%%   \\ \textbf{C}\hspace{0.2in} the end of a particular sprint
%%   \\ \correct{D}\hspace{0.2in}a particular test passes

%% \item \pts{2}
%%   When interviewing for a software engineering role, which of the following are you likely to be evaluated on? Circle all that apply.
%%   \\ \textbf{A}\hspace{0.2in} your actual ability to do the job of a software engineer
%%   \\ \correct{B}\hspace{0.2in}your knowledge of coding and specific algorithms
%%   \\ \correct{C}\hspace{0.2in}your niceness and personality
%%   \\ \textbf{D}\hspace{0.2in} how well you are dressed
  
\item \pts{2}
  \textbf{TRUE} or \correct{FALSE}: continuous integration can only be used by DevOps teams

%% \item \pts{2}
%%   When naming a method, it is a best practice to use a verb-like name if and only if the method has \answershort{side-effects}

\item \pts{2}
  A benefit of modern code review is that more than one person has seen each piece of code that is checked in. This benefit
  reduces your team's \answershort{bus factor}

\item \pts{2}
  A tenet of the \textbf{Waterfall} / \correct{Agile} methodology is to always have a working prototype.

%% \item \pts{2}
%%   The \textbf{Turing machine} and the \textbf{lambda calculus} are two alternative, equivalent formalizations of computability.
%%   Match them with the programming paradigms they inspired: \\
%%   Imperative: \answershort{Turing machine} \\
%%   Functional: \answershort{lambda calculus}

%% \item \pts{4} Give two advantages of static type systems over dynamic type systems and two advantages of dynamic type systems over
%%   static type systems. \\
%%   Advantages of static type systems: \answershort{early detection of errors},\\ \answershort{types are documentation} \\
%%   Advantages of dynamic type systems: \answershort{faster prototyping},\\ \answershort{no false positives}

%%   \item \pts{2} An engineer is working on the backend of a website. He implements a function that makes a rather expensive
%%     database query. When writing test cases, he substitutes a hard-coded string in place of the query. This is an example of
%%      \answershort{mocking}

   \item \pts{2} \correct{TRUE} / \textbf{FALSE}: modern code review is based on an inductive argument for the quality of
     software: if the previous version is good, and the change is good, then modern code review relies on the composition
     of the previous version and the change being good.

   \item \pts{2} You are a manager at MTa (pronounced ``meta''), a social media company for NYC Subway aficionados. One of your
     employees would like to modify an open-source library that is licensed under the GNU Public License, version 2 (GPL v2). You veto
     this decision, because the GPL v2 is a \answershort{copyleft} license: it would force you to release your modifications in
     the open, too.
%\newpage
 %%   \item \pts{2} You are an engineer at Goggle, a tech-focused search company that helps its users find better eyewear. Your team
%%      has been tasked with developing a new microservice that will serve ads to people looking for sunglasses. You and your teammates
%%      gather around a whiteboard and draw a diagram of the new service's \answershort{architecture}

     \newpage
   \item \pts{3} In a single component with a model-view-controller architecture, how many of each of the following should there be?
     \\ Models: \hspace{0.75in} \textbf{0-1} \hspace{0.5in} \correct{exactly 1} \hspace{0.5in} \textbf{1 or more} \hspace{0.5in} \textbf{at least 2}
     \\ Views: \hspace{0.82in} \textbf{0-1} \hspace{0.5in} \textbf{exactly 1} \hspace{0.5in} \correct{1 or more} \hspace{0.5in} \textbf{at least 2}
     \\ Controllers: \hspace{0.51in} \textbf{0-1} \hspace{0.5in} \textbf{exactly 1} \hspace{0.5in} \correct{1 or more} \hspace{0.5in} \textbf{at least 2}

   \item \pts{2} \textbf{TRUE} or \correct{FALSE}: because we have the singleton design pattern, global state is a good design choice

%%    \item \pts{2} When a version control system like \lstinline{git} determines that two changes are conflict-free, which of the following
%%      are possible for the merged code? Circle all that apply.
%%      \\ \correct{A}\hspace{0.2in}compilation errors
%%      \\ \textbf{B}\hspace{0.2in} multiple changes to the same line
%%      \\ \correct{C}\hspace{0.2in}test failures

%%    \item \pts{2} You are an engineer at UTube, a video-streaming platform. You're having trouble with a particularly difficult
%%      programming problem, so you schedule a call with one of your coworkers. Together, you write the difficult code. This is
%%      \answershort{pair programming}

%%    \item \pts{2} Why is exhaustive testing not possible for most programs in practice? Answer in five words or fewer.
%%      \answershort{Input space is too large.}

   \item \pts{2} You are writing a compiler for your new language, Java++ (it's like Java, but with more classes!).
     A user reports that a particular program doesn't compile correctly. You add that program to your test suite,
     and then you start fixing the problem in your compiler. You are practicing
     \\ \answershort{test-driven development}

\item Consider a program with seven sequential if statements that accepts seven boolean
inputs. Assuming each condition evaluates a single unique input, what is the minimum
number of test cases required to achieve:
\begin{enumerate}
\item \pts{1} 100\% branch coverage? \answershort{2: one all false, one all true}
\item \pts{1} 100\% condition coverage? \answershort{14 = 2*7}
\end{enumerate}

     
  \newpage

  \textbf{III. Short answer (\shortpts pts).} Answer the questions in this section in at most two sentences.

%% \item Consider the following code snippet: \\
%%   \begin{lstlisting}
%%     static double computeCircumference(double r) {
%%       return 2 * 3.14 * r;
%%     }
%%   \end{lstlisting}

%%   What are two code-level design improvements that you could make to this method? The two improvements cannot both be the same kind of change.
%%   \begin{enumerate}
%%   \item \pts{3} \answershort{replace 3.14 with a PI constant} 
%%   \item \pts{3} \answershort{rename ``r'' to ``radius'' or rename ``computeCircumference'' to ``circumference''}

%%   \item \pts{3} Select one of your answers to the previous question. Give a one-sentence justification for why your change
%%   improves the method's code-level design.

%%   \answerlong{For 1. above: ``avoid magic numbers'' or ``more self-documenting''. For 2. above: ``radius'' is more descriptive than ``r'', or ``compute'' is a verb, but the method returns a noun-like type.}

%%   \end{enumerate}

%%   \item Suppose that you are a manager at AmiZone, a French social networking company. Radhia, one of your engineers, comes to you
%%     with a proposal to redirect 30\% of the effort you currently spend on integration testing to instead run a static analysis.
%% \begin{enumerate}
%%   \item \pts{3} Describe a situation where Radhia's proposal is a good choice for AmiZone: that is, a situation in which it
%%     is clearly a better allocation of your limited development resources.
%%     \\
%%     \answerlong{Answers can vary. ``AmiZone is concerned about security bugs.'' is the shortest, best answer; short answers that
%%       identify a class of bugs that static analysis is good at will get full credit..}

%%   \item \pts{4} Describe, in one sentence each, two risks associated with Radhia's proposal.\\
%%     \answerlong{Any two of the following sentences would be a good answer (other answers are possible).
%%       Reducing integration testing may increase the risk that bugs at the boundary between modules are not detected.
%%       Static analysis may produce too many false positive alarms.
%%       Static analysis may be unsound or may miss bugs (even of the kind it is supposed to find).
%%       Static analysis errors may be too hard to understand.
%%     }
%%     \end{enumerate}

%\newpage

  \item For each sub-problem, specify if delta debugging can be used
    to solve the problem. If it can, provide a brief description of an "Interesting" function that will help solve the problem. If it
    cannot, specify which properties of delta debugging make it not suitable to solve the problem.
    \begin{enumerate}

    \item \pts{3} Given a list of (positive and negative) integers that sums to zero, we want to find the minimal subset that sums to zero.\\
      \answerlong{Delta debugging is not suitable for this use case: integer summation is ambiguous.}
      \vspace{0.25in}
    \item \pts{3} In an effort to reduce the memory footprint of a Java program, we decide to try replacing
      every \lstinline{long} variable with an \lstinline{int}. However, when we do so, our program fails some of its test cases (we suspect due to integer overflow). We want to
      identify which \lstinline{long} variables can be replaced with \lstinline{int} variables such that the program still passes all tests.\\
      \answerlong{Delta debugging is suitable for this use case. We can define script is_interesting.sh such that it takes a
        list of occurrences of long and replaces each with an int. The script exits 1 if the code compiles and runs the tests
        successfully, and it exits 0 if the code doesn't compile or fails any tests.}
      \vspace{0.25in}
    \item \pts{3} In an effort to reduce the build time of a Java program, we decide to try removing
      dependencies from our build file: the program is old, and we believe some of the dependencies are no longer used.
      However, when we do so, our program fails to compile (we suspect because some of the dependencies we removed are
      still in use). We want to identify the minimal subset of dependencies that enables our program to compile.\\
      \answerlong{Delta debugging is suitable for this use case. We can define script is_interesting.sh such that it takes a list of
        dependencies, removes them from the build file, and then attempts to compile the program. The script exits 1 if the code compiles
        successfully, and 0 if it does not.}
            \vspace{0.25in}
`     \end{enumerate}

%%     \item Consider the following items from the ``Joel Test'', which we read before lecture 2:
%%     \begin{itemize}
%%       \item \textbf{A}: Do you use source control?
%%       \item \textbf{B}: Can you make a build in one step?
%%       \item \textbf{C}: Do you have a bug database?
%%       \item \textbf{D}: Do you fix bugs before writing new code?
%%       \item \textbf{E}: Do you have an up-to-date schedule?
%%     \end{itemize}

%%     \begin{enumerate}
%%     \item \pts{2} Identify one of the above Joel Test items \textbf{A}-\textbf{E} that your group project did better than the other items. Explain why you
%%       were better at this item than the others and how it benefited your project.\\
%%       \answerlong{Answers vary.}

%%     \item \pts{2} Identify one of the above Joel Test items \textbf{A}-\textbf{E} that your group project did worse than the other items. Explain, via
%%       specific examples, how failing to meet this Joel Test requirement impacted your project.\\
%%       \answerlong{Answers vary.}
%%     \end{enumerate}

  \item \pts{4}  Write a method accepting one input parameter for which test input generation
    via constraint solving will work better than test input generation at random.\\
    \answerlong{Answers vary. The key idea is to include a very specific test. For example:\\
      \begin{lstlisting}
        int foo(int x) {
          if (x == 10) return 0;
          else return 1;
        }
      \end{lstlisting}
    }

    \newpage

  \item For each of the following bugs, describe a situation in which the bug would be \emph{high severity}
    and describe another, distinct situation in which the bug would be \emph{low severity}.
    \begin{enumerate}
    \item A programmer-managed resource, such as a socket or database connection, is not closed before
      the last pointer to it goes out of scope.
      \begin{enumerate}
        \item \pts{1} High severity: \answerlong{A long-running program, such as a webserver, which may eventually run out of resources.}
        \item \pts{1} Low severity: \answerlong{a short-running program, such as one that sends a single ping and then exits: the resource will
          be freed when the program ends, anyway. Other examples are possible.}
      \end{enumerate}
    \item An integer value is stored in the wrong units (e.g., feet vs meters, seconds vs hours, etc.).
      \begin{enumerate}
      \item \pts{1} High severity: \answerlong{Mars polar orbiter crash.}
        \item \pts{1} Low severity: \answerlong{The integer value is part of a game
        and is only displayed to the user: e.g., it doesn't matter if an imaginary car's speed is in m/s or mph.
        Other answers are possible.}
          \end{enumerate}
      \end{enumerate}

    \item Consider the following pairs of tools, techniques, or processes. For each pair, give a class of defects or a
situation for which the first element performs better than the second (i.e., is more likely to succeed and reduce software
engineering effort and/or improve software engineering outcomes) and explain why.
\begin{enumerate}
%% \item \pts{3} Agile development model better than waterfall development model\\
%%   \answerlong{Agile is best when requirements are not fully known in advance and the customer is easy to access, because it relies on
%%     fast feedback cycles between your prototypes and the customer.}
\item \pts{3} DevOps approach to operations better than traditional (``sysadmin'') approach to operations\\
  \answerlong{DevOps is best when the organization developing a service is also running that service, becasue operational pain
    is felt by the same org.}
\item \pts{3} Watchpoints better than breakpoints \\
  \answerlong{Watchpoints are better when you know what value shouldn't be changing, but not where in the code that change is occuring.}
%% \item \pts{3} Fuzzing better than regression testing \\
%%   \answerlong{Fuzzing is better if the existing test suite is very small. Another reasonable answer is that fuzzing can find new bugs, but regression testing only prevents bug that have occurred before.}
\item \pts{3} Modern code review better than integration testing \\
  \answerlong{The easiest answer is that code review can find code-design defects, but integration testing cannot. Other answers are possible.}
%% \item \pts{3} Interpreted languages better than compiled languages \\
%%   \answerlong{Interpreted languages are better for rapid prototyping and are usually easier to write code in.}
\end{enumerate}

%% \item You are the customer asking a software company to build you a program that
%%   sorts a list of numbers. You both agree that the output of sort must be in descending order.
%%   The company delivers a program that always returns the empty list. You meet to correct
%%   this. The company then delivers a program that deterministically tries all permutations
%%   of the input list until one is found that is sorted. (This is sometimes called “bogosort”.)
%%   Explain two elements that may have gone wrong during requirements
%%   elicitation (at any stage).
%%   \begin{enumerate}
%%   \item \pts{3} The first element:\\
%%     \answerlong{There was an omission in the functional requirements: the customer failed to state that
%%       the returned list must have the same elements as the input list.}
%%   \item \pts{3} The second element:\\
%%     \answerlong{Quality requirements were not discussed: bogosort is O(n!), which is much slower than
%%       sorting a list ought to be (O(nlogn)). Other well-reasoned answers are also possible.}
%%     \end{enumerate}

\newpage

\item \pts{7} Consider a static analysis that computes, for every program point, which variables contain a value that has
  definitely been used by the program up to that point. (This is a form of \emph{strictness analysis}.) The analysis
  associates an analysis fact with each variable at each program point: either ``T'', meaning ``it is not known whether the
  variable has been used yet'' (aka ``top''); or ``U'', meaning ``the variable has definitely been used at this point.'' The analysis has
  a transfer function for binary addition: after the statement ``x = y + z'', ``y'' and ``z'' are both ``U''. It also has a
  transfer function for ``return x'': after such a statment, ``x'' is ``U''. Other transfer functions are standard for a forwards,
  ``definitely''-style analysis (like the nullness analysis described in class).
  Simulate this analysis on the following program, filling in each blank with the corresponding abstract value (``T'' or ``U'')
  that the analysis would compute for that variable at that point. (The ``*'' boolean operator means ``choose at random'', or equivalently, ``flip a coin''.)
%  (1pt per program point, no partial credit.)

  \begin{lstlisting}[escapechar=\%]
def f(a, b, c) {
                   [ a -> %\answervvshort{T}%, b -> %\answervvshort{T}%, c -> %\answervvshort{T}%, d -> %\answervvshort{T}% ]
  if (*) {
                   [ a -> %\answervvshort{T}%, b -> %\answervvshort{T}%, c -> %\answervvshort{T}%, d -> %\answervvshort{T}% ]
    d = a + b;
                   [ a -> %\answervvshort{U}%, b -> %\answervvshort{U}%, c -> %\answervvshort{T}%, d -> %\answervvshort{T}% ]
  } else {
                   [ a -> %\answervvshort{T}%, b -> %\answervvshort{T}%, c -> %\answervvshort{T}%, d -> %\answervvshort{T}% ]
    d = a + c;
                   [ a -> %\answervvshort{U}%, b -> %\answervvshort{T}%, c -> %\answervvshort{U}%, d -> %\answervvshort{T}% ]
  }
                   [ a -> %\answervvshort{U}%, b -> %\answervvshort{T}%, c -> %\answervvshort{T}%, d -> %\answervvshort{T}% ]
  return d;
                   [ a -> %\answervvshort{U}%, b -> %\answervvshort{T}%, c -> %\answervvshort{T}%, d -> %\answervvshort{U}% ]
}
  \end{lstlisting}

\item \pts{4} You are a software engineer at Orange, a computer hardware company with a software division.
  You are responsible for replacing a microservice that
  determines how much money customers owe to various app store vendors at the end of each month; your service sends this information
  to another service that actually bills the customer. A legacy service exists but takes all month to run. Your boss wants you to rewrite
  it to improve its performance, so that you can compute the customer's bill at any time, not just once a month.
  However, your boss is also concerned about correctness. Describe a low-cost strategy that you
  could apply in this situation to ensure that your new service has the same behavior as the slow, legacy version. Answer in at most 3 sentences.
  \\\answerlong{The correct thing to do here is to apply differential testing (aka ``tests for free''): log all traffic to the old service,
    and check that the new service produces the same answers as the old one for each input for some period of time (until your boss is
    satisfied, probably).}

%% \item \pts{3} Support or refute the claim ``You cannot have continuous integration without hermetic builds.''
%%   Answer in three sentences or fewer.\\
%%   \answerlong{Likely support. Continuous integration does not work well without hermetic builds, because to test in CI you need to be able to build from
%%     scratch.}
  
    \newpage
    \textbf{IV. Document-based Questions (\dbqpts pts).} All questions in this section refer to a documents \textbf{A-B}.
    These documents appear at the end of the exam (I recommend that you tear them out and refer to them as you answer the questions).
    
%%     \item \pts{2} What kind of document is \textbf{Document A}? \\ \answershort{postmortem}

%%     \item \pts{3} What was the root cause of the outage? Answer in the form of a quote
%%       from the document.\\ \answerlong{``one of the inputs to the command was entered incorrectly''}

%%     \item \pts{3} Does the document identify who was responsible for the outage? If so, give a quote
%%       from the document identifying that the responsible party. If not, give a one-sentence explanation
%%       for why not.\\
%%       \answerlong{The document does not identify who was responsible, because this is a blameless postmortem:
%%         it blames the system, rather than the individual. (Any answer with ``blameless'' or a synonym will probably be accepted.)}

%%     \item \pts{3} The root cause of the incident only impacted a single service (S3 itself). Why was the ``blast radius''
%%       of the incident (i.e., the number of impacted customers) so large?\\
%%       \answerlong{Other services relied on S3, so a cascading failure caused them to fail, as well.}
      
%%     \item \pts{3} Describe one change to the tools used to remove capacity that the S3 team could adopt to prevent or mitigate
%%       a similar outage in the future. \\
%%       \answerlong{Either of the following two answers, which S3 itself gives in the full postmortem, is acceptable. Other
%%         sensible answers may also get credit or partial credit. 1) modify the tool to remove capacity more slowly, or
%%         2) add safeguards to prevent capacity from being removed when it will take any subsystem below its minimum required capacity level}

%%     \item \pts{3} Describe another distinct change to the tools used to remove capacity that the S3 team could adopt to prevent
%%       or mitiage a similar outage in the future. \\
%%       \answerlong{See the previous question.}
      
%%     \newpage

%%     Questions on this page refer to \textbf{Document B}, which is a candidate's response to you,
%%     the interviewer, presenting them with the following technical challenge:

%%     \emph{
%%       ``Write isPalindrome(), a function that
%%       returns true if parameter x is a palindrome integer. Note that an integer, like 12321, is a palindrome if it reads the same
%%       forwards and backwards.''
%%     }
      
%%     \textbf{Document B} also includes two questions that the candidate asked you (and your answers) on its first two lines.

%%   \item \label{palindrome:id} \pts{2} Identify two test inputs where the provided \lstinline{isPalidrome()} implementation returns \lstinline{false}.\\
%%     \answerlong{Answers will vary. Student solutions must not be a palindrome. {{123}}, {{10}} are potential answers that return
%%       false.}

%%   \item \pts{3} Your answers to question \ref{palindrome:id} could be which of the following kinds of tests (circle all that apply):
%%     \\ \correct{A}\hspace{0.2in}unit tests
%%     \\ \textbf{B}\hspace{0.2in} integration tests
%%     \\ \textbf{C}\hspace{0.2in} fuzz tests
%%     \\ \correct{D}\hspace{0.2in}partition tests

%%   \item \pts{4}  Identify four things that the candidate did well. (In other words, identify four properties that a company might
%% desire in a software engineer that could potentially be shown by a candidate taking the interview and that were shown by this
%% particular candidate.) Use at most 4 sentences.\\
%% \answerlong{Answers will vary. Potential solutions include the following:
%% 1. The candidate provided inline comments explaining some of their code.
%% 2. The candidate asked relevant questions regarding code functionality.
%% 3. The candidate has consistent indentation.
%% 4. The candidate used a descriptive variable name.}

%% \item \pts{4} \label{lastQuestion} Support or refute the claim that the candidate's implementation of \lstinline{isPalindrome()} is functionally correct. Use
%%   at most 4 sentences.\\
%%   \answerlong{Refute. The candidate's implementation of isPalindrome() is not functionally correct. The candidate missed a base
%% case regarding x being a negative number. Negative integers would not count as a palindrome. {-313} reversed is {313-}.}
    
%%     \newpage

    Questions on this page refer to \textbf{Document A}.
    Assume that the only statements we are interested in are \lstinline{STATEMENT 1} through \lstinline{STATEMENT 4}.

  \item \label{sl1} \pts{2} Write a test suite for \lstinline{foo} with 50\% statement coverage. Express your answer as a list of tuples, e.g., ``(x = \#, y = \#, z = \#), (x = \#, y = \#, z = \#), etc.'', where each ``\#'' is a specific integer. \\
  \answerlong{(x = 6, y = 0, z = 3)}
    
  \item \pts{2} Is it possible to achieve 100\% statement coverage on this method? If so, provide a test suite that has 100\% statement coverage. If not, why not? \\
    \answerlong{It is not possible, because STATEMENT 2 is unreachable.}

  \item \label{el1} Consider the mutation operator ``< to <=''.
    \begin{enumerate}
    \item \pts{2} How many first-order (``mutated only once'') mutants can be produced by applying this mutation operator
    to \lstinline$foo$?\\ \answervshort{2}
    \item \pts{2} What is the mutation score of a test suite containing only the following test input (assume that executing a different statement causes the mutant to be killed):
    (x = 5, y = 2, z = 5)?\\ \answershort{0\%}
    \item \pts{3} Is it possible to change one of x, y, or z in the test input in the previous part of the question to improve the mutation score?
    If so, state which of x, y, and z to change, what its new value should be, and what the new mutation score is. If not, explain
    why not. \\
    \answerlong{z = 6 changes the mutation score to 50\%}
    \item \pts{3} Is it possible to change one of x, y, or z in the test input in the previous part of the question to improve the mutation score again?
    If so, state which of x, y, and z to change, what its new value should be, and what the new mutation score is. If not, explain
    why not. \\
    \answerlong{It is not, because the second mutant is an equivalent mutant.}
    \end{enumerate}

    \newpage

    Questions on this page refer to \textbf{Document B}.

    Suppose that you are a software engineer at \lstinline{codewith.us}, a non-profit that provides computer science education
    services to K-12 students via an online coding platform. Your boss sends you the email in \textbf{Document B}.

  \item \label{lastQuestion} \pts{14} Write a response to your boss. Your response must either provide a reasonable, well-justified plan
    for how you will implement the proposed feature (including task sizing and at least research, implementation, and testing tasks)
    or a well-reasoned justification for why the proposed feature is not feasible.

    \answerlong{The proposed feature is an instance of the halting problem, and Mr. Rice's insistence that
      you get the correct answer 100\% of the time means that it is impossible to implement the feature as described.
    Correct answers must be phrased as an email, and use a polite but firm tone.}

    \newpage
    
    \textbf{V. Extra Credit}. Questions in this section do not count towards the denominator of the exam score.

  \item \ecpts{1} Why did panelist Rupali Vohra (Convoy) say that her undergraduate operating systems class
    was particularly useful to her in her software engineering career?
    \answerlong{Any answer involving the common-ness of distributed systems or the similarity between multithreading and distributed systems is acceptable.}

  \item \ecpts{1} The ``\answershort{campsite}'' approach to managing technical debt, according to our panelists, involves leaving any code you modify
    in less technical debt than your found it in.
  
  \item \ecpts{1} \label{optional1} Name an optional reading assignment that you read but will not and have not used as a response to the ``Optional Reading Response \#1''
    or ``Optional Reading Response \#2'' assignments on Canvas, and then give a one-sentence description of something you learned from that reading.
    (Note: answering this question with a particular reading means you cannot use that reading for your ``Response \#2'', due on May 2.)
    \answerlong{Answers vary.}
    
  \item \ecpts{1} Name another optional reading assignment, distinct from your answer to question \ref{optional1},
    that you read but will not and have not used as a response to the ``Optional Reading Response \#1''
    or ``Optional Reading Response \#2'' assignments on Canvas, and then give a one-sentence description of something you learned from that reading.
    (Note: answering this question with a particular reading means you cannot use that reading for your ``Response \#2'', due on May 2.)
    \answerlong{Answers vary.}
    
  \item \ecpts{1} Name a recent ``hot topic'' in Software Engineering research and give a one-sentence description of that topic.
    \answerlong{Expected answers are the topics covered in class on 20 April, but any topic on which there were 3 or more papers at FSE, ICSE, ISSTA, or ASE in the past 2 years
      is acceptable.}

  \item \ecpts{1} Would you be willing to serve on an engineer panel in the future (i.e., once you have a job)? If so, leave an email address that you expect to still monitor
    after graduating in this space. If not, write ``no'' to receive full credit.\answerlong{``no'' or any email address receives full credit.}
    
    \newpage

    \textbf{Document A:}
    
%%       We’d like to give you some additional information about the service disruption that occurred in the Northern Virginia (US-EAST-1) Region on the morning of February 28th, 2017. The Amazon Simple Storage Service (S3) team was debugging an issue causing the S3 billing system to progress more slowly than expected. At 9:37AM PST, an authorized S3 team member using an established playbook executed a command which was intended to remove a small number of servers for one of the S3 subsystems that is used by the S3 billing process. Unfortunately, one of the inputs to the command was entered incorrectly and a larger set of servers was removed than intended. The servers that were inadvertently removed supported two other S3 subsystems.  One of these subsystems, the index subsystem, manages the metadata and location information of all S3 objects in the region. This subsystem is necessary to serve all GET, LIST, PUT, and DELETE requests. The second subsystem, the placement subsystem, manages allocation of new storage and requires the index subsystem to be functioning properly to correctly operate. The placement subsystem is used during PUT requests to allocate storage for new objects. Removing a significant portion of the capacity caused each of these systems to require a full restart. While these subsystems were being restarted, S3 was unable to service requests. Other AWS services in the US-EAST-1 Region that rely on S3 for storage, including the S3 console, Amazon Elastic Compute Cloud (EC2) new instance launches, Amazon Elastic Block Store (EBS) volumes (when data was needed from a S3 snapshot), and AWS Lambda were also impacted while the S3 APIs were unavailable.

%%       S3 subsystems are designed to support the removal or failure of significant capacity with little or no customer impact. We build our systems with the assumption that things will occasionally fail, and we rely on the ability to remove and replace capacity as one of our core operational processes. While this is an operation that we have relied on to maintain our systems since the launch of S3, we have not completely restarted the index subsystem or the placement subsystem in our larger regions for many years. S3 has experienced massive growth over the last several years and the process of restarting these services and running the necessary safety checks to validate the integrity of the metadata took longer than expected. The index subsystem was the first of the two affected subsystems that needed to be restarted. By 12:26PM PST, the index subsystem had activated enough capacity to begin servicing S3 GET, LIST, and DELETE requests. By 1:18PM PST, the index subsystem was fully recovered and GET, LIST, and DELETE APIs were functioning normally.  The S3 PUT API also required the placement subsystem. The placement subsystem began recovery when the index subsystem was functional and finished recovery at 1:54PM PST. At this point, S3 was operating normally. Other AWS services that were impacted by this event began recovering. Some of these services had accumulated a backlog of work during the S3 disruption and required additional time to fully recover.

%% %      We are making several changes as a result of this operational event. While removal of capacity is a key operational practice, in this instance, the tool used allowed too much capacity to be removed too quickly. We have modified this tool to remove capacity more slowly and added safeguards to prevent capacity from being removed when it will take any subsystem below its minimum required capacity level. This will prevent an incorrect input from triggering a similar event in the future. We are also auditing our other operational tools to ensure we have similar safety checks. We will also make changes to improve the recovery time of key S3 subsystems. We employ multiple techniques to allow our services to recover from any failure quickly. One of the most important involves breaking services into small partitions which we call cells. By factoring services into cells, engineering teams can assess and thoroughly test recovery processes of even the largest service or subsystem. As S3 has scaled, the team has done considerable work to refactor parts of the service into smaller cells to reduce blast radius and improve recovery. During this event, the recovery time of the index subsystem still took longer than we expected. The S3 team had planned further partitioning of the index subsystem later this year. We are reprioritizing that work to begin immediately.

%%       From the beginning of this event until 11:37AM PST, we were unable to update the individual services’ status on the AWS Service Health Dashboard (SHD) because of a dependency the SHD administration console has on Amazon S3. Instead, we used the AWS Twitter feed (@AWSCloud) and SHD banner text to communicate status until we were able to update the individual services’ status on the SHD.  We understand that the SHD provides important visibility to our customers during operational events and we have changed the SHD administration console to run across multiple AWS regions.

%%       Finally, we want to apologize for the impact this event caused for our customers. While we are proud of our long track record of availability with Amazon S3, we know how critical this service is to our customers, their applications and end users, and their businesses. We will do everything we can to learn from this event and use it to improve our availability even further.~
    
%% \vspace{0.5in}
%%     \textbf{Document B:}

%%     \begin{lstlisting}
%% // Q: Can integer x be in range [0-9]? A: Yes.
%% // Q: Should I account for integer overflow? A: Yes.
%% bool isPalindrome(int x) {
%%  // a single digit is a palindrome
%%  if (x < 10) {
%%    return true;
%%  }
%%  // if x's last digit is 0, then its first digit must be 0 in order
%%  // to be a palindrome. In this case, only 0 can be a palindrome.
%%  if (x % 10 == 0 && x != 0) {
%%    return false;
%%  }
%%  // revert the last half of x for comparison against first half
%%  int revertedNumber = 0;
%%  while(x > revertedNumber) {
%%    revertedNumber = revertedNumber * 10 + x % 10;
%%    x /= 10;
%%  }
%%  return x == revertedNumber || x == revertedNumber / 10;
%% }
%% \end{lstlisting}
%    \newpage
%    \newpage
%%     \textbf{Document C:}
    \begin{lstlisting}
    void foo(int x, int y, int z) {
      if (x > 5) {
        // STATEMENT 1
      else if (y < 4 && x > 7) {
        // STATEMENT 2
      }
      if (z < 6) {
        // STATEMENT 3
      } else {
        // STATEMENT 4
      }
    }
    \end{lstlisting}

\vspace{0.5in}
    
    \textbf{Document B:}

    [Your Name],

    I met with the executive board today, and they're onboard with a new feature to help our students avoid
    writing programs that run forever by warning them before they start a program whether it will loop forever.
    I'd like you to design the feature and then write me up an estimate for how much time it'll take you to
    implement. Here are some requirements I gathered from the board:
    \begin{itemize}
    \item All the board members agreed that the warning message needs to be a bright red popup that appears
      when the student is about to run a program that'll run forever.
    \item Ms. Wheat (she used to be a teacher, remember, so she really gets the students!) insisted that we
      use only simple words in the warning message. I think her suggestion was ``Watch out! This might last forever!'', but
      we can iterate on that.
    \item Mr. Rice (who, as you might remember, is our biggest donor - so we have to keep him happy) insisted that
      if we're going to spend technical resources on this and not on his pet project of making all of our buttons larger,
      we really have to get it right every time.
    \item All the board members agreed that you should use a font that really pops out and makes the kids rethink what
      they were about to do. They all still think it's the 90s, though, so we can probably not worry about this too much
      as long as the design you pick gels with the rest of the site.
    \end{itemize}

    Get back to me ASAP with that estimate - I want you working on this next sprint.

    Thanks!
    [Your Boss]
    
    
%% \item What is the value of each expression using the provided
%%   variables. Place a decimal point in your answer to indicate a double
%%   value (eg. 2.0).

%% \begin{lstlisting}
%% double x = 100.0;
%% double y = 15;
%% int m = 25;
%% int n = 10;
%% \end{lstlisting}

%% \begin{enumerate}

%% \item \pts{1} \underline{\hspace{1in}} \lstinline$x / n + m$

%% \item \pts{1} \underline{\hspace{1in}} \lstinline$n / x + m$

%% \item \pts{1} \underline{\hspace{1in}} \lstinline$int(x) / n + m$

%% \item \pts{1} \underline{\hspace{1in}} \lstinline$n / (int)x + m$

%% \item \pts{1} \underline{\hspace{1in}} \lstinline$int(y) % m$

%% \end{enumerate}

%% \item \pts{5} Write a \textbf{templated} function that converts an
%%   array into a string.  For example, if the array contained the
%%   numbers $0,3,4,6$, the expected output would be \lstinline$"[ 0, 3, 4, 6 ]"$. 
%%   You may assume that the type in the array has a
%%   stringstream insertion-operator function defined. (Hint: hw03)

%% \newpage

%% \hspace{-.5cm}\textbf{Questions \ref{sl1}-\ref{el1}} refer to the code shown below. 

%% \begin{lstlisting}[showspaces=false,showlines=true,escapeinside={*@}{@*}]
%% int i, *i1;
%% double* d = new double[5];
%% double* d2 = new double[10];
%% i = 0;
%% i1 = &i;
%% while (i < 5) {
%%   d[i] = i;
%%   *(d2 + (i * 2)) = i;
%%   i = i + 2;
%% }
%% *@\label{firstl1}@*
%% delete [] d;

%% \end{lstlisting}


%% \item\label{sl1} \pts{1} How many variables \textbf{are} declared as
%% pointers? \underline{\hspace{1in}}

%% \item \pts{1} How many variables are \textbf{not} declared as
%% pointers? \underline{\hspace{1in}}

%% \item \pts{2} How many variables have values that are memory locations
%% on the heap at \textbf{line \ref{firstl1}}? \underline{\hspace{1in}}

%% \item \pts{1} Is dynamic memory freed? If so, indicate which lines it
%% occurs. \underline{\hspace{1in}}

%% \item \pts{2} Does the example contain a memory leak? (yes/no)
%% \underline{\hspace{1in}}

%% \item \pts{2} If the example contains a memory leak, show in the space
%%   provided below how it can be corrected without changing the
%%   resulting variable values. If it does not contain a memory leak,
%%   explain why there is none.

%% \vspace{2cm}

%% \item \pts{2} What is the value of \lstinline$*d$ at \textbf{line
%%   \ref{firstl1}}? \underline{\hspace{1in}}

%% \item \pts{2} What is the value of \lstinline$*i1$ at \textbf{line
%%   \ref{firstl1}}? \underline{\hspace{1in}}

%% \item \pts{2} What is the value of \lstinline$i1$ at \textbf{line
%%   \ref{firstl1}}? \underline{\hspace{1in}}

%% \item \pts{2} What is the value of \lstinline$d[2]$ at \textbf{line
%%   \ref{firstl1}}? \underline{\hspace{1in}}

%% \item\label{el1} \pts{2} What is the value of \lstinline$d2[2]$ at \textbf{line
%%   \ref{firstl1}}? \underline{\hspace{1in}}

%% \newpage

%% \hspace{-.5cm}\textbf{Questions \ref{rl1}-\ref{rel1}} refer to the code shown below. 

%% \begin{lstlisting}[showspaces=false,showlines=true]
%% int f(int n) {
%%   if (n == 0)
%%     return 1;
%%   else
%%     return n * f(n-1);
%% }
%% \end{lstlisting}

%% \item\label{rl1} \pts{2} Show the result of evaluating the function
%%   \textbf{f} for the values: 0, 2, 3, and 5.

%% \vspace{4cm}

%% \item \pts{1} The line number of the return statement for the base
%%   case in the function \textbf{f} is \underline{\hspace{1in}}.

%% \item\label{rel1} \pts{1} The line number for the recursive case in
%%   the function is \underline{\hspace{1in}}.


%% \hspace{-.5cm}Use the following listing to answer the question below.

%% \begin{multicols}{2}
%% \begin{lstlisting}[numbers=none,frame=none]
%% class Node {
%% public:
%%   int x;
%%   Node* next;
%% };
%% class Stack {
%%   Node* top;
%% public:
%%   void push(int val);
%% };
%% \end{lstlisting}
%% \end{multicols}


%% \item \pts{4} Convert the \lstinline$Node$ and \lstinline$Stack$ class
%%   specifications, as shown above, into templated classes.


%% \vspace{7cm}

%% \newpage

%% \item\label{ab} \pts{10} Assuming that the classes A and B have been defined as
%%   shown below, what does the following program display as output?
%%   (Write output on lines below.) 

%% \begin{multicols}{2}
%% \begin{lstlisting}[numbers=none,frame=none]
%% class A {
%%   public:
%%   virtual string m() const {
%%     return "A";
%%   }
%% };

%% class B: public A {
%%   public:
%%   virtual string m() const {
%%     return "B";
%%   }
%% };
%% \end{lstlisting}
%% \end{multicols}
%% \begin{lstlisting}[escapeinside={*@}{@*}]
%% void f1(A a) { cout << a.m() << endl; }

%% void f2(A& a) { cout << a.m() << endl; } 

%% int main() {
%%   A* a = new A();
%%   B* b = new B();

%%   cout << a->m() << endl;*@\label{l1}@*
%%   cout << b->m() << endl;*@\label{l3}@*
%%   a = b;
%%   cout << a->m() << endl;*@\label{l5}@*
  
%%   f1(*a);*@\label{l7}@*
%%   f2(*a);*@\label{l8}@*

%%   return 0; 
%% }
%% \end{lstlisting}

%% \newcommand{\lin}[1]{
%% \par\smallskip\noindent\parbox[t]{.09\textwidth}{\raggedright\textbf{Line #1:}}
%%  \parbox[t]{.3\textwidth}{\raggedleft\hrulefill}\par\smallskip\vspace{1em}
%% }%

%% \begin{multicols}{2}

%% \lin{\ref{l1}}

%% \lin{\ref{l3}}

%% \lin{\ref{l5}}

%% \lin{\ref{l7}}

%% \lin{\ref{l8}}

%% \end{multicols}

%% \newpage

%% \begin{lstlisting}
%% A* a = new B();
%% B* b = dynamic_cast<B*>(a);

%% if (b == NULL)
%%   cout << "I'm NOT a B object?" << endl;
%% else
%%   cout << "I'm a B object!" << endl;
%% \end{lstlisting}

%% \item \pts{2} Assuming the definitions from question \ref{ab}, write
%%   the output from the code shown above.

%% \vspace{2cm}

%% \item \pts{4} Depict the binary search tree (BST) generated by
%% inserting the following keys: 3, 1, 6, 4, 10, 2, 5, 9, 8, 7. You need
%% only show the final tree.

%% \vspace{6cm}

%% \item\label{fbst} \pts{3} Depict the BST after removing 8, 6, and
%% 1. Depict the BST after each removal is completed. Use the removal
%% method used in class and in the book.

%% \vspace{12cm}

%% \item \pts{2} List the \textbf{preorder} traversal of the final tree
%% in question \ref{fbst}.

%% \vspace{2cm}

%% \item \pts{2} List the \textbf{postorder} traversal of the final tree
%% in question \ref{fbst}.

%% \vspace{2cm}

%% \item \pts{2} List the \textbf{inorder} traversal of the final tree in
%% question \ref{fbst}.

%% \vspace{2cm}

%% \item \pts{2} List the \textbf{breadth-first} traversal of the tree in
%% question \ref{fbst}.

%% \vspace{2cm}

%% \item\label{tree} \pts{3} For each node in the following tree,
%%   write the height of the subtree rooted at its node to the left and
%%   write the AVL balance factor to the right.

%% \begin{center}

%% \tikzset{
%%   tnode/.style = {text centered, font=\sffamily, circle, black, draw=black, inner sep=2pt, minimum width=2em}
%% }


%% \resizebox{12.0cm}{!}{%
%% \begin{tikzpicture}[level/.style={sibling distance = 16cm/(2^#1),
%%   level distance = 2cm}] 

%% \node [tnode] { 18 }
%%   child { node [tnode] { 10 } 
%%     child { node [tnode] { 5 } 
%%       child { node [tnode] { 2 } }
%%       child[missing]{}
%%     }
%%     child { node [tnode] { 15 } 
%%       child { node [tnode] { 13 } }
%%       child[missing]{}
%%     }
%%   } 
%%   child { node [tnode] { 23 }
%%     child { node [tnode] { 20 } }
%%     child[missing]{} 
%%   }
%% ;
%% \end{tikzpicture}
%% }%
%% \end{center}

%% \newpage

%% \item \pts{2} Is the tree of question \ref{tree} a \textbf{binary heap}?

%% \vspace{1cm}

%% \item \pts{2} Is the tree of question \ref{tree} a \textbf{binary search tree}?

%% \vspace{1cm}

%% \item \pts{2} Is the tree of question \ref{tree} an \textbf{AVL tree}?

%% \vspace{1cm}


%% \item \pts{8} Depict the \textbf{AVL tree} generated by inserting the
%%   following keys: 3, 1, 6, 4, 10, 2, 5, 9, 8, 7. Clearly label and
%%   depict each rotation with the proper rotation from the AVL cheat
%%   sheet (LL,RR,etc.).  You will lose significant points if rotations
%%   are not clearly marked.

%% \newpage

%% \item \pts{5} Show the action of removing 10, 7, and 9
%%   from the final AVL tree of the previous question.  Depict the AVL
%%   tree after each removal is completed and label any rotations. Use
%%   the removal method used in class and in the book.

%% \vspace{6cm}


%% \item\label{splaytree} \pts{8} Depict the \textbf{splay tree} generated by
%%   inserting the following keys into an initially empty tree: 10, 25,
%%   1, 14, 17, 86, 15, 2, 6, 13. Clearly label any splay operations that
%%   occur.

%% \newpage

%% \begin{center}
%% \textit{Additional space for the previous question}
%% \end{center}

%% \vspace{8cm}

%% \item\label{lastQuestion} \pts{5} Show the action of performing
%%   searches on the final splay tree of the previous question for keys
%%   17, 14, and 2.

%% \newpage

%% \textbf{Extra Credit}: \ecpts{5} Depict all non-isomorphic \textbf{binary
%% search trees} containing keys from the set ${1,2,3,4}$.  If the number
%% of trees is too large to depict, explain how to calculate the number
%% of trees.

%% \vspace{12cm}

%% \textbf{Extra Credit}: \ecpts{5} Depict all non-isomorphic \textbf{AVL trees}
%% containing keys from the set ${1,2,3,4}$.  If the number of trees is
%% too large to depict, explain how to calculate the number of trees.


\end{enumerate}
\end{spacing}

\end{document}


