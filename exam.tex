\documentclass{article}
% Change "article" to "report" to get rid of page number on title page
% \usepackage{amsmath,amsfonts,amsthm,amssymb}
\usepackage{setspace}
\usepackage{fancyhdr}
\usepackage{lastpage}
\usepackage{extramarks}
\usepackage{textcomp}
\usepackage{amsmath}
\usepackage{lstcustom}
\usepackage{enumitem}
\usepackage{multicol}
\usepackage{url}
\usepackage{tikz}

\usepackage{totcount} % For the total points

\usepackage{xspace}

% In case you need to adjust margins:
\topmargin=-0.45in
\evensidemargin=0in
\oddsidemargin=0in
\textwidth=6.5in
\textheight=9.0in
\headsep=0.25in

% Commands for answer key
\newif\ifkey
%\keytrue
\keyfalse

\newcommand{\correct}[1]{\ifkey\color{red}\textbf{#1}\color{black}\else\textbf{#1}\fi\xspace}
\newcommand{\answershort}[1]{\ifkey\color{red}\underline{\textbf{#1}}\color{black}\else\underline{\hspace{3in}}\fi\xspace}

% Commands for points
\newtotcounter{points}
\newcommand*{\totalpoints}{\thepoints}
\newcommand*{\pts}[1]{\addtocounter{points}{#1}(#1pt)}

\newtotcounter{ecpoints}
\newcommand*{\totalecpoints}{\theecpoints}
\newcommand{\ecpts}[1]{\addtocounter{ecpoints}{#1}(#1pt)}

% Exercise Specific Information
\newcommand{\hmwkTitle}{Final Exam}
\newcommand{\hmwkDueDate}{UCID:\underline{\hspace{1in}}}
\newcommand{\hmwkClass}{CS 490} \newcommand{\hmwkPoints}{
  \ref{lastQuestion} questions; \protect\total{points} pts +
  \protect{\total{ecpoints}} ec; \pageref{LastPage} pgs.}

\newcommand{\cstart}{\vspace{.4cm}}

% Setup the header and footer
\pagestyle{fancy}
\lhead{\hmwkClass\ \hmwkTitle} 
\chead{\hmwkPoints}
\rhead{\hmwkDueDate}
\lfoot{\lastxmark}  
\cfoot{}            
\rfoot{Page\ \thepage\ of\ \pageref{LastPage}}
\renewcommand\headrulewidth{0.4pt} 
\renewcommand\footrulewidth{0.4pt}

% Setup listing look

\lstset{
  language=C++,
  style=eclipse,
  showspaces=false, 
  numbers=left,
  frame=tb,
}

\begin{document}

\begin{spacing}{1.4}

\begin{enumerate}[leftmargin=*]
\item \pts{1} \textbf{Name:} \hrulefill

Carefully read each question, and write the answer in the space
provided.  If answers to free response questions are written obscurely,
zero credit will be awarded. The correct answer to a free response question
will never contain any significant words used in the question itself (i.e., ``crossword rules'').
You are permitted one 8.5x11 inch sheet of paper (double-sided)
containing notes; all other aids (other than your brain) are forbidden.
Questions may be brought to the instructor.

For \textbf{TRUE} or \textbf{FALSE} and multiple choice questions,
circle your answer.

On free response questions only, you will receive \textbf{20\%} credit
for any question which you leave blank (i.e., do not attempt to
answer). Do not waste your time or mine by making up an answer if you
do not know. (Note though that most questions offer partial credit, so
if you know part of the answer, it is almost always better to write something
rather than nothing.)

To get credit for this question, you must:
\begin{itemize}
\item Print your name (e.g., ``Martin Kellogg'') in the space provided on this page.
\item Print your UCID (e.g., ``mjk76'') in the space at the top of \textbf{each} page of the exam.
\end{itemize}
  
\newpage

\textbf{Reading Quiz Redux (5pts)}

\item \pts{1}
  \textbf{Static Analysis, Part 2:} \textbf{TRUE} or
  \correct{FALSE}: to use the verifier, engineers were taught how to
  use a special, declarative programming language that was not similar
  to their regular development language (C). The author's ICSE paper
  reports on how easy it was to teach this language to C developers.

\item \pts{1}
  \textbf{Debugging, Part 2:} \correct{TRUE} or \textbf{FALSE}:
  delta debugging requires a test to prove that each circumstance is really failure inducing.

\item \pts{1}
  \textbf{Code-level Design:} Name an advantage of \texttt{black} over the other Python linters discussed in the Yelp whitepaper. (< 5 words)
  \\ \answershort{any of: opinionated; resolves errors automatically; consistency}

\item \pts{1}
  \textbf{Tech Debt, Part 1:} \textbf{TRUE} or \correct{FALSE}:
  all technical debt is the result of programmer laziness.

\item \pts{1}
  \textbf{Tech Debt, Part 2:}
  The author claims that most programmers, when asked about the system they’re working on, “think the old code is a mess”. He posits this is due to a “fundamental law of programming”. Which one?
  \\ \correct{A}\hspace{0.2in}reading code is harder than writing code
  \\ \textbf{B}\hspace{0.2in} the halting problem
  \\ \textbf{C}\hspace{0.2in} given enough eyeballs, all bugs are shallow

\item \pts{1}
  \textbf{Requirements, Part 2:}
  The author describes formal specifications as providing three main benefits. Which of the following is NOT one of those:
  \\ \textbf{A}\hspace{0.2in} It provides clear documentation of the system requirements, behavior, and properties.
  \\ \textbf{B}\hspace{0.2in} It clarifies your understanding of the system.
  \\ \textbf{C}\hspace{0.2in} It finds really subtle, dangerous bugs.
  \\ \correct{D}\hspace{0.2in}It makes writing the code quicker and easier.

\item \pts{1}
  \textbf{Free and Open Source Software:}
  The author claims that the term “free software” means:
  \\ \textbf{A}\hspace{0.2in} software you can get for zero price
  \\ \correct{B}\hspace{0.2in}software which gives the user certain freedoms
  \\ \textbf{C}\hspace{0.2in} software whose source code you can look at
  \\ \textbf{D}\hspace{0.2in} none of the above

\item \pts{1}
  \textbf{Software Architecture, Part 1:}
  The author argues that which of the following should drive the design of a software system’s architecture:
  \\ \textbf{A}\hspace{0.2in} the existing implementation
  \\ \textbf{B}\hspace{0.2in} a set of guidelines from an architecture book
  \\ \correct{C}\hspace{0.2in}the system’s quality requirements

\item \pts{1}
  \textbf{Static Analysis, Part 1:}
  FindBugs \underline{\hspace{1in}}:
  \\ \textbf{A}\hspace{0.2in} always warns about line X if it is possible there is a bug on line X
  \\ \textbf{B}\hspace{0.2in} never warns about line X unless there is definitely a bug on line X
  \\ \textbf{C}\hspace{0.2in} both A and B
  \\ \correct{D}\hspace{0.2in}neither A nor B

\item \pts{1}
  \textbf{DevOps, Part 2:}
  TODO

  \newpage

  \textbf{Multiple Choice and Very Short Answer (X pts).} In the following section, either circle your
  answer (possible answers appear in \textbf{bold}) or write a very short (one word or one phrase) answer in the space provided.

\item \pts{2} Google (and other big tech companies) design their hiring process
  to avoid false \correct{positive} / \textbf{negative} results: that is, to avoid hiring unqualified candidates,
  even if some good candidates are rejected.

\item \pts{2} A \textbf{sound} / \correct{complete} program analysis always answers ``I don't know'' unless there
  is definitely a bug in the program being analyzed.

\item \pts{2} A \correct{functional specification} / \textbf{quality requirement} is a description of what a system should do that doesn’t specify how the system should do it.

\item \pts{2} Alice's deadline to deliver a feature is today. She finishes writing the feature and tests it on her local machine, but chooses not to write automated tests,
  even though she knows that it is risky, because
  she wants to meet her deadline. Alice has chosen to take on \answershort{technical debt}

\item \pts{2}
  Which of the following is NOT a static analysis:
  \\ \textbf{A}\hspace{0.2in} dataflow analysis
  \\ \correct{B}\hspace{0.2in}testing
  \\ \textbf{C}\hspace{0.2in} code review

\item \pts{2}
  Which of the following is it best practice to commit to your version control system? Circle all that apply.
  \\ \textbf{A}\hspace{0.2in} credentials
  \\ \correct{B}\hspace{0.2in}code
  \\ \textbf{C}\hspace{0.2in} binary files
  \\ \correct{D}\hspace{0.2in}config files

\item \pts{2}
  \textbf{TRUE} or \correct{FALSE}: continuous integration can only be used by DevOps teams

\item \pts{2}
  When naming a method, it is a best practice to use a verb-like name if and only if the method has \answershort{side-effects}
  
%% \item What is the value of each expression using the provided
%%   variables. Place a decimal point in your answer to indicate a double
%%   value (eg. 2.0).

%% \begin{lstlisting}
%% double x = 100.0;
%% double y = 15;
%% int m = 25;
%% int n = 10;
%% \end{lstlisting}

%% \begin{enumerate}

%% \item \pts{1} \underline{\hspace{1in}} \lstinline$x / n + m$

%% \item \pts{1} \underline{\hspace{1in}} \lstinline$n / x + m$

%% \item \pts{1} \underline{\hspace{1in}} \lstinline$int(x) / n + m$

%% \item \pts{1} \underline{\hspace{1in}} \lstinline$n / (int)x + m$

%% \item \pts{1} \underline{\hspace{1in}} \lstinline$int(y) % m$

%% \end{enumerate}

%% \item \pts{5} Write a \textbf{templated} function that converts an
%%   array into a string.  For example, if the array contained the
%%   numbers $0,3,4,6$, the expected output would be \lstinline$"[ 0, 3, 4, 6 ]"$. 
%%   You may assume that the type in the array has a
%%   stringstream insertion-operator function defined. (Hint: hw03)

%% \newpage

%% \hspace{-.5cm}\textbf{Questions \ref{sl1}-\ref{el1}} refer to the code shown below. 

%% \begin{lstlisting}[showspaces=false,showlines=true,escapeinside={*@}{@*}]
%% int i, *i1;
%% double* d = new double[5];
%% double* d2 = new double[10];
%% i = 0;
%% i1 = &i;
%% while (i < 5) {
%%   d[i] = i;
%%   *(d2 + (i * 2)) = i;
%%   i = i + 2;
%% }
%% *@\label{firstl1}@*
%% delete [] d;

%% \end{lstlisting}


%% \item\label{sl1} \pts{1} How many variables \textbf{are} declared as
%% pointers? \underline{\hspace{1in}}

%% \item \pts{1} How many variables are \textbf{not} declared as
%% pointers? \underline{\hspace{1in}}

%% \item \pts{2} How many variables have values that are memory locations
%% on the heap at \textbf{line \ref{firstl1}}? \underline{\hspace{1in}}

%% \item \pts{1} Is dynamic memory freed? If so, indicate which lines it
%% occurs. \underline{\hspace{1in}}

%% \item \pts{2} Does the example contain a memory leak? (yes/no)
%% \underline{\hspace{1in}}

%% \item \pts{2} If the example contains a memory leak, show in the space
%%   provided below how it can be corrected without changing the
%%   resulting variable values. If it does not contain a memory leak,
%%   explain why there is none.

%% \vspace{2cm}

%% \item \pts{2} What is the value of \lstinline$*d$ at \textbf{line
%%   \ref{firstl1}}? \underline{\hspace{1in}}

%% \item \pts{2} What is the value of \lstinline$*i1$ at \textbf{line
%%   \ref{firstl1}}? \underline{\hspace{1in}}

%% \item \pts{2} What is the value of \lstinline$i1$ at \textbf{line
%%   \ref{firstl1}}? \underline{\hspace{1in}}

%% \item \pts{2} What is the value of \lstinline$d[2]$ at \textbf{line
%%   \ref{firstl1}}? \underline{\hspace{1in}}

%% \item\label{el1} \pts{2} What is the value of \lstinline$d2[2]$ at \textbf{line
%%   \ref{firstl1}}? \underline{\hspace{1in}}

%% \newpage

%% \hspace{-.5cm}\textbf{Questions \ref{rl1}-\ref{rel1}} refer to the code shown below. 

%% \begin{lstlisting}[showspaces=false,showlines=true]
%% int f(int n) {
%%   if (n == 0)
%%     return 1;
%%   else
%%     return n * f(n-1);
%% }
%% \end{lstlisting}

%% \item\label{rl1} \pts{2} Show the result of evaluating the function
%%   \textbf{f} for the values: 0, 2, 3, and 5.

%% \vspace{4cm}

%% \item \pts{1} The line number of the return statement for the base
%%   case in the function \textbf{f} is \underline{\hspace{1in}}.

%% \item\label{rel1} \pts{1} The line number for the recursive case in
%%   the function is \underline{\hspace{1in}}.


%% \hspace{-.5cm}Use the following listing to answer the question below.

%% \begin{multicols}{2}
%% \begin{lstlisting}[numbers=none,frame=none]
%% class Node {
%% public:
%%   int x;
%%   Node* next;
%% };
%% class Stack {
%%   Node* top;
%% public:
%%   void push(int val);
%% };
%% \end{lstlisting}
%% \end{multicols}


%% \item \pts{4} Convert the \lstinline$Node$ and \lstinline$Stack$ class
%%   specifications, as shown above, into templated classes.


%% \vspace{7cm}

%% \newpage

%% \item\label{ab} \pts{10} Assuming that the classes A and B have been defined as
%%   shown below, what does the following program display as output?
%%   (Write output on lines below.) 

%% \begin{multicols}{2}
%% \begin{lstlisting}[numbers=none,frame=none]
%% class A {
%%   public:
%%   virtual string m() const {
%%     return "A";
%%   }
%% };

%% class B: public A {
%%   public:
%%   virtual string m() const {
%%     return "B";
%%   }
%% };
%% \end{lstlisting}
%% \end{multicols}
%% \begin{lstlisting}[escapeinside={*@}{@*}]
%% void f1(A a) { cout << a.m() << endl; }

%% void f2(A& a) { cout << a.m() << endl; } 

%% int main() {
%%   A* a = new A();
%%   B* b = new B();

%%   cout << a->m() << endl;*@\label{l1}@*
%%   cout << b->m() << endl;*@\label{l3}@*
%%   a = b;
%%   cout << a->m() << endl;*@\label{l5}@*
  
%%   f1(*a);*@\label{l7}@*
%%   f2(*a);*@\label{l8}@*

%%   return 0; 
%% }
%% \end{lstlisting}

%% \newcommand{\lin}[1]{
%% \par\smallskip\noindent\parbox[t]{.09\textwidth}{\raggedright\textbf{Line #1:}}
%%  \parbox[t]{.3\textwidth}{\raggedleft\hrulefill}\par\smallskip\vspace{1em}
%% }%

%% \begin{multicols}{2}

%% \lin{\ref{l1}}

%% \lin{\ref{l3}}

%% \lin{\ref{l5}}

%% \lin{\ref{l7}}

%% \lin{\ref{l8}}

%% \end{multicols}

%% \newpage

%% \begin{lstlisting}
%% A* a = new B();
%% B* b = dynamic_cast<B*>(a);

%% if (b == NULL)
%%   cout << "I'm NOT a B object?" << endl;
%% else
%%   cout << "I'm a B object!" << endl;
%% \end{lstlisting}

%% \item \pts{2} Assuming the definitions from question \ref{ab}, write
%%   the output from the code shown above.

%% \vspace{2cm}

%% \item \pts{4} Depict the binary search tree (BST) generated by
%% inserting the following keys: 3, 1, 6, 4, 10, 2, 5, 9, 8, 7. You need
%% only show the final tree.

%% \vspace{6cm}

%% \item\label{fbst} \pts{3} Depict the BST after removing 8, 6, and
%% 1. Depict the BST after each removal is completed. Use the removal
%% method used in class and in the book.

%% \vspace{12cm}

%% \item \pts{2} List the \textbf{preorder} traversal of the final tree
%% in question \ref{fbst}.

%% \vspace{2cm}

%% \item \pts{2} List the \textbf{postorder} traversal of the final tree
%% in question \ref{fbst}.

%% \vspace{2cm}

%% \item \pts{2} List the \textbf{inorder} traversal of the final tree in
%% question \ref{fbst}.

%% \vspace{2cm}

%% \item \pts{2} List the \textbf{breadth-first} traversal of the tree in
%% question \ref{fbst}.

%% \vspace{2cm}

%% \item\label{tree} \pts{3} For each node in the following tree,
%%   write the height of the subtree rooted at its node to the left and
%%   write the AVL balance factor to the right.

%% \begin{center}

%% \tikzset{
%%   tnode/.style = {text centered, font=\sffamily, circle, black, draw=black, inner sep=2pt, minimum width=2em}
%% }


%% \resizebox{12.0cm}{!}{%
%% \begin{tikzpicture}[level/.style={sibling distance = 16cm/(2^#1),
%%   level distance = 2cm}] 

%% \node [tnode] { 18 }
%%   child { node [tnode] { 10 } 
%%     child { node [tnode] { 5 } 
%%       child { node [tnode] { 2 } }
%%       child[missing]{}
%%     }
%%     child { node [tnode] { 15 } 
%%       child { node [tnode] { 13 } }
%%       child[missing]{}
%%     }
%%   } 
%%   child { node [tnode] { 23 }
%%     child { node [tnode] { 20 } }
%%     child[missing]{} 
%%   }
%% ;
%% \end{tikzpicture}
%% }%
%% \end{center}

%% \newpage

%% \item \pts{2} Is the tree of question \ref{tree} a \textbf{binary heap}?

%% \vspace{1cm}

%% \item \pts{2} Is the tree of question \ref{tree} a \textbf{binary search tree}?

%% \vspace{1cm}

%% \item \pts{2} Is the tree of question \ref{tree} an \textbf{AVL tree}?

%% \vspace{1cm}


%% \item \pts{8} Depict the \textbf{AVL tree} generated by inserting the
%%   following keys: 3, 1, 6, 4, 10, 2, 5, 9, 8, 7. Clearly label and
%%   depict each rotation with the proper rotation from the AVL cheat
%%   sheet (LL,RR,etc.).  You will lose significant points if rotations
%%   are not clearly marked.

%% \newpage

%% \item \pts{5} Show the action of removing 10, 7, and 9
%%   from the final AVL tree of the previous question.  Depict the AVL
%%   tree after each removal is completed and label any rotations. Use
%%   the removal method used in class and in the book.

%% \vspace{6cm}


%% \item\label{splaytree} \pts{8} Depict the \textbf{splay tree} generated by
%%   inserting the following keys into an initially empty tree: 10, 25,
%%   1, 14, 17, 86, 15, 2, 6, 13. Clearly label any splay operations that
%%   occur.

%% \newpage

%% \begin{center}
%% \textit{Additional space for the previous question}
%% \end{center}

%% \vspace{8cm}

%% \item\label{lastQuestion} \pts{5} Show the action of performing
%%   searches on the final splay tree of the previous question for keys
%%   17, 14, and 2.

%% \newpage

%% \textbf{Extra Credit}: \ecpts{5} Depict all non-isomorphic \textbf{binary
%% search trees} containing keys from the set ${1,2,3,4}$.  If the number
%% of trees is too large to depict, explain how to calculate the number
%% of trees.

%% \vspace{12cm}

%% \textbf{Extra Credit}: \ecpts{5} Depict all non-isomorphic \textbf{AVL trees}
%% containing keys from the set ${1,2,3,4}$.  If the number of trees is
%% too large to depict, explain how to calculate the number of trees.


\end{enumerate}
\end{spacing}

\end{document}


